\documentclass[journal,12pt,twocolumn]{IEEEtran}
\usepackage{amsthm}
\allowbreak
\usepackage{setspace}
\usepackage{gensymb}
\singlespacing
\usepackage[cmex10]{amsmath}
\usepackage{caption}
\usepackage{amsthm}

\DeclareUnicodeCharacter{2212}{-}
\usepackage{tikz}
\usepackage{pgfplots}

\usepackage{mathrsfs}
\usepackage{txfonts}
\usepackage{stfloats}
\usepackage{bm}
\usepackage{cite}
\usepackage{cases}
\usepackage{subfig}

\usepackage{longtable}
\usepackage{multirow}

\usepackage{enumitem}
\usepackage{mathtools}
\usepackage{steinmetz}
\usepackage{tikz}
\usepackage{circuitikz}
\usepackage{verbatim}
\usepackage{tfrupee}
\usepackage[breaklinks=true]{hyperref}
\usepackage{graphicx}
\usepackage{tkz-euclide}
\graphicspath{ {./images/} }
\usetikzlibrary{calc,math}
\usepackage{listings}
    \usepackage{color}                                            %%
    \usepackage{array}                                            %%
    \usepackage{longtable}                                        %%
    \usepackage{calc}                                             %%
    \usepackage{multirow}                                         %%
    \usepackage{hhline}                                           %%
    \usepackage{ifthen}                                           %%
    \usepackage{lscape}     
\usepackage{multicol}
\usepackage{chngcntr}

\DeclareMathOperator*{\Res}{Res}

\newcommand{\comb}[2]{{}^{#1}\mathrm{C}_{#2}}

\renewcommand\thesection{\arabic{section}}
\renewcommand\thesubsection{\thesection.\arabic{subsection}}
\renewcommand\thesubsubsection{\thesubsection.\arabic{subsubsection}}

\renewcommand\thesectiondis{\arabic{section}}
\renewcommand\thesubsectiondis{\thesectiondis.\arabic{subsection}}
\renewcommand\thesubsubsectiondis{\thesubsectiondis.\arabic{subsubsection}}


\hyphenation{op-tical net-works semi-conduc-tor}
\def\inputGnumericTable{}                                 %%

\lstset{
%language=C,
frame=single, 
breaklines=true,
columns=fullflexible
}
\begin{document}


\newtheorem{theorem}{Theorem}[section]
\newtheorem{problem}{Problem}
\newtheorem{proposition}{Proposition}[section]
\newtheorem{lemma}{Lemma}[section]
\newtheorem{corollary}[theorem]{Corollary}
\newtheorem{example}{Example}[section]
\newtheorem{definition}[problem]{Definition}

\newcommand{\BEQA}{\begin{eqnarray}}
\newcommand{\EEQA}{\end{eqnarray}}
\newcommand{\define}{\stackrel{\triangle}{=}}
\bibliographystyle{IEEEtran}
\raggedbottom
\setlength{\parindent}{0pt}
\providecommand{\mbf}{\mathbf}
\providecommand{\pr}[1]{\ensuremath{\Pr\left(#1\right)}}
\providecommand{\qfunc}[1]{\ensuremath{Q\left(#1\right)}}
\providecommand{\sbrak}[1]{\ensuremath{{}\left[#1\right]}}
\providecommand{\lsbrak}[1]{\ensuremath{{}\left[#1\right.}}
\providecommand{\rsbrak}[1]{\ensuremath{{}\left.#1\right]}}
\providecommand{\brak}[1]{\ensuremath{\left(#1\right)}}
\providecommand{\lbrak}[1]{\ensuremath{\left(#1\right.}}
\providecommand{\rbrak}[1]{\ensuremath{\left.#1\right)}}
\providecommand{\cbrak}[1]{\ensuremath{\left\{#1\right\}}}
\providecommand{\lcbrak}[1]{\ensuremath{\left\{#1\right.}}
\providecommand{\rcbrak}[1]{\ensuremath{\left.#1\right\}}}
\theoremstyle{remark}
\newtheorem{rem}{Remark}
\newcommand{\sgn}{\mathop{\mathrm{sgn}}}
\providecommand{\abs}[1]{$\left\vert#1\right\vert$}
\providecommand{\res}[1]{\Res\displaylimits_{#1}} 
\providecommand{\norm}[1]{$\left\lVert#1\right\rVert$}
%\providecommand{\norm}[1]{\lVert#1\rVert}
\providecommand{\mtx}[1]{\mathbf{#1}}
\providecommand{\mean}[1]{E$\left[ #1 \right]$}
\providecommand{\fourier}{\overset{\mathcal{F}}{ \rightleftharpoons}}
%\providecommand{\hilbert}{\overset{\mathcal{H}}{ \rightleftharpoons}}
\providecommand{\system}{\overset{\mathcal{H}}{ \longleftrightarrow}}
	%\newcommand{\solution}[2]{\textbf{Solution:}{#1}}
\newcommand{\solution}{\noindent \textbf{Solution: }}
\newcommand{\cosec}{\,\text{cosec}\,}
\providecommand{\dec}[2]{\ensuremath{\overset{#1}{\underset{#2}{\gtrless}}}}
\newcommand{\myvec}[1]{\ensuremath{\begin{pmatrix}#1\end{pmatrix}}}
\newcommand{\mydet}[1]{\ensuremath{\begin{vmatrix}#1\end{vmatrix}}}
\numberwithin{equation}{subsection}
\makeatletter
\@addtoreset{figure}{problem}
\makeatother
\let\StandardTheFigure\thefigure
\let\vec\mathbf
\renewcommand{\thefigure}{\theproblem}
\def\putbox#1#2#3{\makebox[0in][l]{\makebox[#1][l]{}\raisebox{\baselineskip}[0in][0in]{\raisebox{#2}[0in][0in]{#3}}}}
     \def\rightbox#1{\makebox[0in][r]{#1}}
     \def\centbox#1{\makebox[0in]{#1}}
     \def\topbox#1{\raisebox{-\baselineskip}[0in][0in]{#1}}
     \def\midbox#1{\raisebox{-0.5\baselineskip}[0in][0in]{#1}}
\vspace{3cm}
\title{Assignment 4}
\author{Vijay Varma - AI20BTECH11012}
\maketitle
\newpage
\bigskip
\renewcommand{\thefigure}{\theenumi}
\renewcommand{\thetable}{\theenumi}

%
Download latex-tikz codes from 
%
\begin{lstlisting}
https://github.com/KBVijayVarma/AI1103-Assignment-3
\end{lstlisting}
\section*{\textbf{Problem GATE 2017 (CS-SET 2), Q.60}}
There are 3 red socks, 4 green socks and 3 blue socks. You choose 2 socks. The probability that they are of the same colour is 
\begin{multicols}{4}
\begin{enumerate}
    \item $\frac{1}{5}$
    \item $\frac{7}{30}$
    \item $\frac{1}{4}$ 
    \item $\frac{4}{15}$ 
\end{enumerate}
\end{multicols}
\section*{\textbf{Solution}}
Let $X_1 \in \{0,1,2\}$ and $X_2 \in \{0,1,2\}$ be two Random Variables representing the colour of socks taken in $1^{st}$ draw and in $2^{nd}$ draw respectively.

$X_1 = 0$, $X_1 = 1$, $X_1 = 2$ represent choosing Red, Green, Blue socks in the first draw respectively.

Similarly, $X_2 = 0$, $X_2 = 1$, $X_2 = 2$ represent choosing Red, Green, Blue socks in the second draw respectively.

Now, the probability that the socks drawn in $1^{st}$ draw and $2^{nd}$ draw are of the same colour is given by
\begin{align*}
\pr{X_1 = X_2}    
\end{align*}
Now,
\begin{align}
\pr{X_1 = X_2} &= \sum_{k = 0}^{k = 2} \pr{X_1 = X_2 = k} \\
&= \sum_{k = 0}^{k = 2} \pr{X_1 = k, X_2 = k} \\
&= \sum_{k = 0}^{k = 2} \pr{X_1 = k} \pr{(X_2 = k) | (X_1 = k)}
\end{align}
\begin{align}
&= \pr{X_1 = 0} \pr{(X_2 = 0) | (X_1 = 0)} \\ \nonumber
& \quad + \pr{X_1 = 1} \pr{(X_2 = 1) | (X_1 = 1)} \\ \nonumber
& \quad + \pr{X_1 = 2} \pr{(X_2 = 2) | (X_1 = 2)}
\end{align}
From the given information in the question,
\begin{align}
\pr{X_1 = X_2} &= \left(\frac{3}{10} \right) \left(\frac{2}{9} \right) + \left(\frac{4}{10} \right) \left(\frac{3}{9} \right) + \left(\frac{3}{10} \right) \left(\frac{2}{9} \right) \\
&= \left(\frac{6}{90} \right) + \left(\frac{12}{90} \right) + \left(\frac{6}{90} \right) \\
&= \frac{24}{90} = \frac{4}{15}
\end{align}
Therefore, the probability that the two socks are of same colour is $\frac{4}{15}$.

Hence, the correct option is 4) $\frac{4}{15}$.


\end{document}